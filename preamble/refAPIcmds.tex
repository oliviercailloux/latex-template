\usepackage{xstring}
\usepackage{pgffor}

\ExplSyntaxOn
\NewDocumentCommand{\extractlast}{O{.}mo}
 {
  \seq_set_split:Nnn \l_stroobants_string_seq { #1 } { #2 }
  \IfNoValueTF { #3 }
   {
    \seq_item:Nn \l_stroobants_string_seq { -1 }
   }
   {
    \tl_set:Nx #3 { \seq_item:Nn \l_stroobants_string_seq { -1 } }
   }
 }
\seq_new:N \l_stroobants_string_seq
\ExplSyntaxOff

\newbool{refAPIFormatTypewriter}
%show at sign if present in input
\newbool{refAPIShowAt}
\newbool{refAPIShowPackage}
\newbool{refAPIShowField}
\newbool{refAPIShowMethod}
\newbool{refAPIShowParameters}
\pgfkeys{
	/refAPI/.is family,
	/refAPI/.cd,
	display string/.estore in = \refAPIDisplayString,
	formatTypewriter/.is if = refAPIFormatTypewriter,
	showAt/.is if = refAPIShowAt,
	showPackage/.is if = refAPIShowPackage,
	showField/.is if = refAPIShowField,
	showMethod/.is if = refAPIShowMethod,
	showParameters/.is if = refAPIShowParameters,
	default/.style = {
		display string =,
		formatTypewriter = true,
		showAt = true,
		showPackage = false,
		showField = true,
		showMethod = true,
		showParameters = false,
	},
}

%Thx https://tex.stackexchange.com/a/34318
%TODO URL without module?
%TODO Refine into MethodWithParameters, MethodWithParametersWithoutSpaces and MethodWithoutParameters.
%Approach: define the following refAPI macros. AtSign (starts with at); Module (before unique / or empty); Method (after unique # if has ( or empty); Field (after unique # if no ( or empty if no #); FQName (between / and # or after / or before # or everything); FQNameSlashes (FQName with slashes instead of dots); Class (between last dot and #).
%OLD - Defines: \refAPIFQName (before #) ; \refAPIFieldFull (after #) ; \refAPIFieldShort (after # and before -) ; \refAPIPackageSlashes (FQName before / with slashes instead of dots) ; \refAPIClass (FQName after /)
%{java.base/java.time.zone.ZoneOffsetTransition} ⇒ https://docs.oracle.com/en/java/javase/11/docs/api/java.base/java/time/zone/ZoneOffsetTransition.html
%{java.base/java.time.zone.ZoneOffsetTransition\#getDateTimeAfter()} ⇒ https://docs.oracle.com/en/java/javase/11/docs/api/java.base/java/time/zone/ZoneOffsetTransition.html#getDateTimeAfter()
%{java.base/java.time.zone.ZoneOffsetTransition\#compareTo(java.time.zone.ZoneOffsetTransition)} ⇒ https://docs.oracle.com/en/java/javase/11/docs/api/java.base/java/time/zone/ZoneOffsetTransition.html#compareTo(java.time.zone.ZoneOffsetTransition)
%{java.base/java.lang.String\#substring(int, int)} ⇒ https://docs.oracle.com/en/java/javase/11/docs/api/java.base/java/lang/String.html#substring(int,int) NB need to drop space in parameter
%{@java.base/java.lang.annotation.Documented} ⇒ https://docs.oracle.com/en/java/javase/11/docs/api/java.base/java/lang/annotation/Documented.html
%{@java.base/java.lang.SuppressWarnings\#value()} ⇒ https://docs.oracle.com/en/java/javase/11/docs/api/java.base/java/lang/SuppressWarnings.html#value()
%The main argument specifies everything needed to build the URL and the default appearance (@ or no @…), except possibly for default global variables such as base url. From this, a default appearance is built, which can be changed with keys. Must be able to: hide @ sign; hide method; hide parameters; hide package; replace everything with provided display string; display in texttt mode or not?.
%Shows at iff showAt AND main argument starts with @
\newcommand{\jrefAPI}[2][]{%
	\pgfkeys{/refAPI/.cd, default, #1}%
	\IfBeginWith{#2}{@}{%
		\edef\refAPIAtSign{@}%
		\StrBehind{#2}{@}[\refAPIWithoutAt]%
	}{%
		\edef\refAPIAtSign{}%
		\edef\refAPIWithoutAt{#2}%
	}%
	\IfSubStr{#2}{/}{%
		\StrBefore{\refAPIWithoutAt}{/}[\refAPIModule]%
	}{%
		\edef\refAPIModule{}%
	}%
	\IfSubStr{#2}{\#}{%
		\StrBehind{#2}{\#}[\refAPIAfterSharp]%
		\IfSubStr{\refAPIAfterSharp}{(}{%
			\StrBehind{#2}{\#}[\refAPIMethodWithParameters]%
		}{%
			\edef\refAPIMethodWithParameters{}%
		}%
	}{%
		\edef\refAPIMethodWithParameters{}%
	}%
	\StrSubstitute{\refAPIMethodWithParameters}{ }{}[\refAPIMethodWithParametersWithoutSpaces]%
	\StrBefore{\refAPIMethodWithParameters}{(}[\refAPIMethodWithoutParameters]%
	\IfSubStr{#2}{\#}{%
		\StrBehind{#2}{\#}[\refAPIAfterSharp]%
		\IfSubStr{\refAPIAfterSharp}{(}{%
			\edef\refAPIField{}%
		}{%
			\StrBehind{#2}{\#}[\refAPIField]%
		}%
	}{%
		\edef\refAPIField{}%
	}%
	\IfSubStr{#2}{/}{%
		\StrBehind{#2}{/}[\refAPIAfterSlash]%
		\IfSubStr{\refAPIAfterSlash}{\#}{%
			\StrBefore{\refAPIAfterSlash}{\#}[\refAPIFQName]%
		}{%
			\edef\refAPIFQName{\refAPIAfterSlash}%
		}%
	}{%
		\IfSubStr{#2}{\#}{%
			\StrBefore{\refAPIWithoutAt}{\#}[\refAPIFQName]%
		}{%
			\edef\refAPIFQName{\refAPIWithoutAt}%
		}%
	}%
	\StrSubstitute{\refAPIFQName}{.}{/}[\refAPIFQNameSlashes]%
	\IfSubStr{#2}{\#}{%
		\StrBefore{\refAPIWithoutAt}{\#}[\refAPIBeforeSharp]%
		\edef\refAPIClass{\extractlast{\refAPIBeforeSharp}}%
	}{%
		\edef\refAPIClass{\extractlast{#2}}%
	}%
	%
	%\refAPIPrintAll%
	%
	\edef\refAPITarget{\refAPIBaseUrl}%
	\IfEq{\refAPIModule}{}{%
	}{%
		\appto\refAPITarget{\refAPIModule}%
		\appto\refAPITarget{/}%
	}%
	\appto\refAPITarget{\refAPIFQNameSlashes}%
	\appto\refAPITarget{.html}%
	\IfEq{\refAPIField}{}{%
	}{%
		\appto\refAPITarget{\#\refAPIField}%
	}%
	\IfEq{\refAPIMethodWithParametersWithoutSpaces}{}{%
	}{%
		\IfEq{\refAPIParametersStyle}{hyphen}{%
			\StrSubstitute{\refAPIMethodWithParametersWithoutSpaces}{(}{-}[\refAPIMethodWithParametersWithoutSpacesWithHyphens]%
			\StrSubstitute{\refAPIMethodWithParametersWithoutSpacesWithHyphens}{,}{-}[\refAPIMethodWithParametersWithoutSpacesWithHyphens]%
			\StrSubstitute{\refAPIMethodWithParametersWithoutSpacesWithHyphens}{)}{-}[\refAPIMethodWithParametersWithoutSpacesWithHyphens]%
			\appto\refAPITarget{\#\refAPIMethodWithParametersWithoutSpacesWithHyphens}%
		}{%
			\appto\refAPITarget{\#\refAPIMethodWithParametersWithoutSpaces}%
		}%
	}%
	%
	\edef\refAPILinkText{}%
	\ifbool{refAPIShowAt}{%
		\appto\refAPILinkText{\refAPIAtSign}%
	}{%
	}%
	\ifbool{refAPIShowPackage}{%
		\appto\refAPILinkText{\refAPIFQName}%
	}{%
		\appto\refAPILinkText{\refAPIClass}%
	}%
	\ifbool{refAPIShowField}{%
		\IfEq{\refAPIField}{}{%
		}{%
			\appto\refAPILinkText{\#\refAPIField}%
		}%
	}{%
	}%
	\ifbool{refAPIShowMethod}{%
		\ifbool{refAPIShowParameters}{%
			\IfEq{\refAPIMethodWithParameters}{}{%
			}{%
				\appto\refAPILinkText{\#\refAPIMethodWithParameters}%
			}%
		}{%
			\IfEq{\refAPIMethodWithoutParameters}{}{%
			}{%
				\appto\refAPILinkText{\#\refAPIMethodWithoutParameters}%
			}%
		}%
	}{%
	}%
	%	
	\href{\refAPITarget}{\refAPILinkText}%
}

\newcommand{\refAPIPrintAll}{%
	\begin{description}
		\item[BaseUrl] \foreach \sfx in {BaseUrl}{\csname refAPI\sfx\endcsname}
		\foreach \sfx in {AtSign, Module, MethodWithParameters, MethodWithParametersWithoutSpaces, MethodWithoutParameters, Field, FQName, FQNameSlashes, Class, ParametersStyle}{\item[\sfx] \csname refAPI\sfx\endcsname}
	\end{description}%
}

\DeclareExpandableDocumentCommand{\refAPIBaseUrl}{}{https://docs.oracle.com/javase/8/docs/api/}
\DeclareExpandableDocumentCommand{\refAPIParametersStyle}{}{hyphen}
%\DeclareExpandableDocumentCommand{\refAPIBaseUrl}{}{https://docs.oracle.com/en/java/javase/12/docs/api/}
%\DeclareExpandableDocumentCommand{\refAPIParametersStyle}{}{parenthesis}
%\newcommand{\jeeref}[2][]{\jrefAPI[#1, base url = https://docs.oracle.com/javaee/7/api/]{#2}}
\newcommand{\jseref}[2][]{\jrefAPI[#1]{#2}}
%Standard HTML RenderKit : https://docs.oracle.com/javaee/7/javaserver-faces-2-2/renderkitdocs/HTML_BASIC/javax.faces.Inputjavax.faces.Text.html
%\newcommand{\jsftag}[2][]{\jrefAPI{base url = https://docs.oracle.com/javaee/7/javaserver-faces-2-2/vdldocs-facelets/, full, prefix=:, #1}}


